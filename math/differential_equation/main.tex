\documentclass{jlreq}

\usepackage{titlesec}
\usepackage{listings}
\usepackage{fancyhdr}

% \adjustbox
\usepackage{adjustbox}

% tcolorboxの設定
\usepackage[most]{tcolorbox} 
\tcbuselibrary{breakable}
\tcbuselibrary{skins}
\tcbuselibrary{listingsutf8}
% タイトルのフォーマットを変更
\titleformat{\title}
  {\centering\Huge\bfseries}
  {}
  {0em} 
  {}

\titleformat{\subtitle}
  {\centering\Large\itshape}
  {}
  {0em}
  {}

\titleformat{\subsubsection}[block]
  {\normalfont\normalsize\bfseries}
  {\arabic{subsubsection}.}
  {1em}
  {}

\titleformat{\section}[block]
  {\normalfont\large\bfseries}
  {\Roman{section}.}
  {1em} 
  {}
  [\titleline{\titlerule[1pt]}]

\titleformat{\subsection}[block]
  {\normalfont\normalsize\bfseries}
  {\roman{subsection}.}
  {1em}
  {}

% listingsの設定

\renewcommand{\lstlistingname}{コード}

\lstset{
	breaklines = true,
	language = Python,
	keywordstyle = {\bfseries \color[cmyk]{0,1,0,0}},
	commentstyle = {\itshape \color[cmyk]{1,0.4,1,0}},
	numbers = left,
	numberstyle = \tiny,
	stepnumber = 1,
	% frameとnumberの間の距離
	numbersep = 10pt,
	frame = single,
	basicstyle = \ttfamily,
	tabsize = 2,
	captionpos = t,
	backgroundcolor={\color[gray]{.90}},
	showstringspaces = false,
}

% headerの設定
\pagestyle{fancy}
\fancyhf{}

\fancyhead[RO,RE]{\rightmark}
\fancyhead[LO,LE]{\leftmark} 
\fancyfoot[C]{\thepage}

% tikzの設定
\usepackage{tikz}

\begin{document}
\section{変数分離法}
変数分離形の微分方程式は次の形をしている。

\begin{equation}
  \frac{dy}{dx} = f(x)g(y)
\end{equation}

\subsection{解法}
\begin{enumerate}
  \item 式(1)を$g(y)$で割る
  \item 両辺に$dx$をかける
  \item 両辺に$\int$をつけて積分する
\end{enumerate}

\subsection{例題}

\begin{tcolorbox}[enhanced,
  colback=white!85!gray,
  drop fuzzy shadow,
  boxrule=0.3mm,
  arc=0mm,
  left=0pt,
  top=0pt,
  sharp corners,
  width=\textwidth,
  ]
  \textbf{問題 1}

以下の微分方程式を解け。

(1) $\frac{d}{dx} y = 2 xy$
\tcblower

\begin{tcolorbox}[
  coltext=white!10!blue,
  colback=white!90!purple!90!blue,
  drop fuzzy shadow,
  boxrule=0mm,
  arc=0mm,
  width=1.3cm,
  left=0pt,
  right=0pt,
  top=0pt,
  bottom=0pt,
  halign=flush left,
]
\end{tcolorbox}
\tcblower
\textbf{回答}

(1) \\
解法に従って解くと、$y \neq 0$のとき

\begin{align*}
  \frac{dy}{dx} &= 2xy \\
  \frac{dy}{y} &= 2xdx \\
  \int \frac{dy}{y} &= \int 2xdx \\
  \log |y| &= x^2 + C \\
  y &= \pm e^{x^2 + C} \\
  y &= Ce^{x^2}
\end{align*}

ここで$y = 0$の場合も微分方程式を満たし、$C = 0$とすればいいので、$C$は任意定数である。

\end{tcolorbox}
\end{document}
