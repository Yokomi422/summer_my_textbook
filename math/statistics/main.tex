\documentclass{jlreq}

\usepackage{titlesec}
\usepackage{listings}
\usepackage{fancyhdr}

% \adjustbox
\usepackage{adjustbox}

% tcolorboxの設定
\usepackage[most]{tcolorbox} 
\tcbuselibrary{breakable}
\tcbuselibrary{skins}
\tcbuselibrary{listingsutf8}
% タイトルのフォーマットを変更
\titleformat{\title}
  {\centering\Huge\bfseries}
  {}
  {0em} 
  {}

\titleformat{\subtitle}
  {\centering\Large\itshape}
  {}
  {0em}
  {}

\titleformat{\subsubsection}[block]
  {\normalfont\normalsize\bfseries}
  {\arabic{subsubsection}.}
  {1em}
  {}

\titleformat{\section}[block]
  {\normalfont\large\bfseries}
  {\Roman{section}.}
  {1em} 
  {}
  [\titleline{\titlerule[1pt]}]

\titleformat{\subsection}[block]
  {\normalfont\normalsize\bfseries}
  {\roman{subsection}.}
  {1em}
  {}

% listingsの設定

\renewcommand{\lstlistingname}{コード}

\lstset{
	breaklines = true,
	language = Python,
	keywordstyle = {\bfseries \color[cmyk]{0,1,0,0}},
	commentstyle = {\itshape \color[cmyk]{1,0.4,1,0}},
	numbers = left,
	numberstyle = \tiny,
	stepnumber = 1,
	% frameとnumberの間の距離
	numbersep = 10pt,
	frame = single,
	basicstyle = \ttfamily,
	tabsize = 2,
	captionpos = t,
	backgroundcolor={\color[gray]{.90}},
	showstringspaces = false,
}

% headerの設定
\pagestyle{fancy}
\fancyhf{}

\fancyhead[RO,RE]{\rightmark}
\fancyhead[LO,LE]{\leftmark} 
\fancyfoot[C]{\thepage}

% tikzの設定
\usepackage{tikz}

\begin{document}
\section{用語の確認}
\textbf{1. 確率質量関数と確率密度関数} \\
確率変数には離散型と連続型があり、確率変数の出やすさを表す関数をそれぞれ、\textbf{確率質量関数 (Probability mass function)}、\textbf{確率密度関数 (Probability density function)}
という。

\vspace{0.5cm}
\begin{center}
\begin{tabular}{c|c}
  \textbf{確率質量関数 (PMF)} & \textbf{確率密度関数 (PDF)} \\
  \begin{tikzpicture}[scale=0.8]
      \draw[->] (0,0) -- (5,0) node[right] {$x$};
      \draw[->] (0,0) -- (0,4) node[above] {$P(X = x)$};
      
      \foreach \x/\y in {1/1.5, 2/2.5, 3/1, 4/3} {
          \draw[fill=blue] (\x,0) -- (\x,\y) circle (3pt);
      }
      
      \node[below] at (1,0) {$x_1$};
      \node[below] at (2,0) {$x_2$};
      \node[below] at (3,0) {$x_3$};
      \node[below] at (4,0) {$x_4$};
  \end{tikzpicture}
  &
  \begin{tikzpicture}[scale=0.8]
      \draw[->] (0,0) -- (5,0) node[right] {$x$};
      \draw[->] (0,0) -- (0,4) node[above] {$f(x)$};
      
      \draw[thick, domain=0:5, samples=100, smooth] plot (\x,{3*exp(-0.5*(\x-2.5)^2)});
      
      \node[below] at (2.5,0) {$\mu$};
  \end{tikzpicture}
\end{tabular}
\end{center}

PMFとPDFは確率を表しているため全区間で確率を足すと1になる。

\begin{itemize}
  \item PMF: $\sum_{i} P(X = x_i) = 1$
  \item PDF: $\int_{-\infty}^{\infty} f(x) dx = 1$
\end{itemize}

\textbf{2. 確率変数と実現値} \\
\textbf{確率変数}とは確率的に起こる事柄に実数を対応させたものであり、\textbf{実現値}は実際に観測された値である。確率変数は
大文字で表し、実現値は小文字で表す。確率変数$X$が$X = x$となったとき、$X$は起こりうる値の中で$x$になったということを表す。

\textbf{3. モーメント母関数} \\
\textbf{4. 期待値} \\
\textbf{5. 分散} \\

\section{確率分布}
\section{大数の法則}


\end{document}
