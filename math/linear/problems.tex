\documentclass{jlreq}

\usepackage{titlesec}
\usepackage{listings}
\usepackage{fancyhdr}

% \adjustbox
\usepackage{adjustbox}

% 数学
\usepackage{amssymb} 

% tcolorboxの設定
\usepackage[most]{tcolorbox} 
\tcbuselibrary{breakable}
\tcbuselibrary{skins}
\tcbuselibrary{listingsutf8}
% タイトルのフォーマットを変更
\titleformat{\title}
  {\centering\Huge\bfseries}
  {}
  {0em} 
  {}

\titleformat{\subtitle}
  {\centering\Large\itshape}
  {}
  {0em}
  {}

\titleformat{\subsubsection}[block]
  {\normalfont\normalsize\bfseries}
  {\arabic{subsubsection}.}
  {1em}
  {}

\titleformat{\section}[block]
  {\normalfont\large\bfseries}
  {\Roman{section}.}
  {1em} 
  {}
  [\titleline{\titlerule[1pt]}]

\titleformat{\subsection}[block]
  {\normalfont\normalsize\bfseries}
  {\roman{subsection}.}
  {1em}
  {}

% listingsの設定

\renewcommand{\lstlistingname}{コード}

\lstset{
	breaklines = true,
	language = Python,
	keywordstyle = {\bfseries \color[cmyk]{0,1,0,0}},
	commentstyle = {\itshape \color[cmyk]{1,0.4,1,0}},
	numbers = left,
	numberstyle = \tiny,
	stepnumber = 1,
	% frameとnumberの間の距離
	numbersep = 10pt,
	frame = single,
	basicstyle = \ttfamily,
	tabsize = 2,
	captionpos = t,
	backgroundcolor={\color[gray]{.90}},
	showstringspaces = false,
}

% headerの設定
\pagestyle{fancy}
\fancyhf{}

\fancyhead[RO,RE]{\rightmark}
\fancyhead[LO,LE]{\leftmark} 
\fancyfoot[C]{\thepage}

% tikzの設定
\usepackage{tikz}

% 定理環境
\newtcolorbox{theorembox}[1][]{
    enhanced,
    colback=white!95!green,
    colframe=green!40!black,
    coltitle=black,
    fonttitle=\bfseries,
    title=#1,
    attach boxed title to top left={yshift=-2mm, xshift=2mm},
    boxed title style={colback=green!30!white, size=small},
    drop fuzzy shadow,
    boxrule=0.5mm,
    sharp corners,
    top=4mm, bottom=4mm,
}

% 定義用のボックス環境
\newtcolorbox{definitionbox}[1][]{
    enhanced,
    title=#1, 
    attach boxed title to top left, 
    colback=white!95!blue,
    colbacktitle=white!10!blue!50!black,
    drop fuzzy shadow,
    boxrule=0.25mm,
}

% 問題用のボックス環境
\newtcolorbox{problem}[1][]{enhanced,
  colback=white!85!gray,
  drop fuzzy shadow,
  boxrule=0.3mm,
  arc=0mm,
  left=0pt,
  top=0pt,
  sharp corners,
  width=\textwidth,
  title=\textbf{問題},
  #1
}

\begin{document}
\section{線形代数と有名不等式}
\subsection{三角不等式}
\begin{theorembox}[三角不等式]
	$\boldsymbol{x}, \boldsymbol{y} \in \mathbb{R}^n$に対し、次の不等式が成り立つ。
	\begin{enumerate}
		\item  $\| \boldsymbol{x} \| - \| \boldsymbol{y} \| \leq \| \boldsymbol{x} + \boldsymbol{y} \| \leq \| \boldsymbol{x} \| + \| \boldsymbol{y} \|$
		\item $\| \boldsymbol{x} - \boldsymbol{y} \| \geq | \| \boldsymbol{x} \| - \| \boldsymbol{y} \| |$
	\end{enumerate}
\end{theorembox}

三角不等式を利用した問題を数問解いてみよう。

\begin{problem}
	次の2つの式を満たす平面ベクトル$\boldsymbol{x} = (x, y)$を考える。
	\begin{equation*}
		|\boldsymbol{x} + 2\boldsymbol{y}| = 1
	\end{equation*}	
	\begin{equation*}
		|2\boldsymbol{x} + \boldsymbol{y}| = 1
	\end{equation*}
	このとき、$|\boldsymbol{x} - 3 \boldsymbol{y}|$の最大値と最小値を求めよ。
\end{problem}

\begin{problem}
	関数$f(t) = \sqrt{t^2 + 1} + \sqrt{t^2 - 2t + 1}$ ($0 \leq t \leq 1$)が最小値をとる$t$の値を求めよ。
\end{problem}

\begin{problem}
	関数$g(t) = \sqrt{t^2 + 1}  \sqrt{t^2 - 2 t + 2}$ ($t > 1$)の最大値とそのときの$t$の値を求めよ。
\end{problem}

\subsection{コーシー・シュワルツの不等式}

\begin{theorembox}[コーシー・シュワルツの不等式]
	$\boldsymbol{x}, \boldsymbol{y} \in \mathbb{R}^n$に対し、次の不等式が成り立つ。
	\begin{equation*}
		| \boldsymbol{x} \cdot \boldsymbol{y} | \leq \| \boldsymbol{x} \| \| \boldsymbol{y} \|
	\end{equation*}
\end{theorembox}

コーシー・シュワルツの不等式を利用した問題を数問解いてみよう。

\begin{problem}
	$x, y, z > 0$, $x + y + z = 1$のとき、以下に答えよ。
	\begin{enumerate}
		\item $x^2 + y^2 + z^2$の最小値
		\item $\frac{1}{x} + \frac{1}{y} + \frac{1}{z}$の最小値
	\end{enumerate}
\end{problem}

\end{document}