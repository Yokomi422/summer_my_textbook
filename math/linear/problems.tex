\documentclass{jlreq}

\usepackage{titlesec}
\usepackage{listings}
\usepackage{fancyhdr}

\usepackage{url}

% \adjustbox
\usepackage{adjustbox}

% 数学
\usepackage{amssymb} 

% tcolorboxの設定
\usepackage[most]{tcolorbox} 
\tcbuselibrary{breakable}
\tcbuselibrary{skins}
\tcbuselibrary{listingsutf8}
% タイトルのフォーマットを変更
\titleformat{\title}
  {\centering\Huge\bfseries}
  {}
  {0em} 
  {}

\titleformat{\subtitle}
  {\centering\Large\itshape}
  {}
  {0em}
  {}

\titleformat{\subsubsection}[block]
  {\normalfont\normalsize\bfseries}
  {\arabic{subsubsection}.}
  {1em}
  {}

\titleformat{\section}[block]
  {\normalfont\large\bfseries}
  {\Roman{section}.}
  {1em} 
  {}
  [\titleline{\titlerule[1pt]}]

\titleformat{\subsection}[block]
  {\normalfont\normalsize\bfseries}
  {\roman{subsection}.}
  {1em}
  {}

% listingsの設定

\renewcommand{\lstlistingname}{コード}

\lstset{
	breaklines = true,
	language = Python,
	keywordstyle = {\bfseries \color[cmyk]{0,1,0,0}},
	commentstyle = {\itshape \color[cmyk]{1,0.4,1,0}},
	numbers = left,
	numberstyle = \tiny,
	stepnumber = 1,
	% frameとnumberの間の距離
	numbersep = 10pt,
	frame = single,
	basicstyle = \ttfamily,
	tabsize = 2,
	captionpos = t,
	backgroundcolor={\color[gray]{.90}},
	showstringspaces = false,
}

% headerの設定
\pagestyle{fancy}
\fancyhf{}

\fancyhead[RO,RE]{\rightmark}
\fancyhead[LO,LE]{\leftmark} 
\fancyfoot[C]{\thepage}

% tikzの設定
\usepackage{tikz}

% 定理環境
\newtcolorbox{theorembox}[1][]{
    enhanced,
    colback=white!95!green,
    colframe=green!40!black,
    coltitle=black,
    fonttitle=\bfseries,
    title=#1,
    attach boxed title to top left={yshift=-2mm, xshift=2mm},
    boxed title style={colback=green!30!white, size=small},
    drop fuzzy shadow,
    boxrule=0.5mm,
    sharp corners,
    top=4mm, bottom=4mm,
}

% 定義用のボックス環境
\newtcolorbox{definitionbox}[1][]{
    enhanced,
    title=#1, 
    attach boxed title to top left, 
    colback=white!95!blue,
    colbacktitle=white!10!blue!50!black,
    drop fuzzy shadow,
    boxrule=0.25mm,
}

% 問題用のボックス環境
\newtcolorbox{problem}[1][]{enhanced,
  colback=white!85!gray,
  drop fuzzy shadow,
  boxrule=0.3mm,
  arc=0mm,
  left=0pt,
  top=0pt,
  sharp corners,
  width=\textwidth,
  title=\textbf{問題},
  #1
}

\begin{document}
\section{線形代数と有名不等式}
\subsection{三角不等式}
\begin{theorembox}[三角不等式]
	$\boldsymbol{x}, \boldsymbol{y} \in \mathbb{R}^n$に対し、次の不等式が成り立つ。
	\begin{enumerate}
		\item  $\| \boldsymbol{x} \| - \| \boldsymbol{y} \| \leq \| \boldsymbol{x} + \boldsymbol{y} \| \leq \| \boldsymbol{x} \| + \| \boldsymbol{y} \|$
		\item $\| \boldsymbol{x} - \boldsymbol{y} \| \geq | \| \boldsymbol{x} \| - \| \boldsymbol{y} \| |$
	\end{enumerate}
\end{theorembox}

三角不等式を利用した問題を数問解いてみよう。

\begin{problem}
	次の2つの式を満たす平面ベクトル$\boldsymbol{x} = (x, y)$を考える。
	\begin{equation*}
		|\boldsymbol{x} + 2\boldsymbol{y}| = 1
	\end{equation*}	
	\begin{equation*}
		|2\boldsymbol{x} + \boldsymbol{y}| = 1
	\end{equation*}
	このとき、$|\boldsymbol{x} - 3 \boldsymbol{y}|$の最大値と最小値を求めよ。
\end{problem}

\begin{problem}
	関数$f(t) = \sqrt{t^2 + 1} + \sqrt{t^2 - 2t + 1}$ ($0 \leq t \leq 1$)が最小値をとる$t$の値を求めよ。
\end{problem}

\begin{problem}
	関数$g(t) = \sqrt{t^2 + 1}  \sqrt{t^2 - 2 t + 2}$ ($t > 1$)の最大値とそのときの$t$の値を求めよ。
\end{problem}

\subsection{コーシー・シュワルツの不等式}

\begin{theorembox}[コーシー・シュワルツの不等式]
	$\boldsymbol{x}, \boldsymbol{y} \in \mathbb{R}^n$に対し、次の不等式が成り立つ。
	\begin{equation*}
		| \boldsymbol{x} \cdot \boldsymbol{y} | \leq \| \boldsymbol{x} \| \| \boldsymbol{y} \|
	\end{equation*}
\end{theorembox}

コーシー・シュワルツの不等式を利用した問題を数問解いてみよう。

\begin{problem}
	$x, y, z > 0$, $x + y + z = 1$のとき、以下に答えよ。
	\begin{enumerate}
		\item $x^2 + y^2 + z^2$の最小値
		\item $\frac{1}{x} + \frac{1}{y} + \frac{1}{z}$の最小値
	\end{enumerate}
\end{problem}

参考
\begin{itemize}
  \item \url{https://www.chart.co.jp/subject/sugaku/suken_tsushin/76/76-3.pdf}
  \item \url{https://mathematicsgarden.com/cschwarzprac/}
\end{itemize}

\section{座標空間と数ベクトル空間}
ここでは2次元と3次元の座標空間の問題を解いてみよう。

\begin{theorembox}[直線のパラメータ表示]
  直線$l$上の1点$\boldsymbol{p} = (x_0, y_0, z_0)$と方向ベクトル$\boldsymbol{a} = (a, b, c)$が与えられたとき、直線$l$上の任意の点$\boldsymbol{r} = (x, y, z)$は次のように表される。

  \begin{equation*}
    \begin{pmatrix}
      x \\ y \\ z 
    \end{pmatrix}
    = \boldsymbol{p} + t \boldsymbol{a} = 
    \begin{pmatrix}
      x_0 + t a \\ y_0 + t b \\ z_0 + t c
    \end{pmatrix}
  \end{equation*}
  パラメータ$t$を消去すると、直線$l$の方程式は次のように表される。

  \begin{equation*}
    \frac{x - x_0}{a} = \frac{y - y_0}{b} = \frac{z - z_0}{c}
  \end{equation*}
\end{theorembox}

\begin{theorembox}[平面の方程式]
  平面上の点$\boldsymbol{p} = (x_0, y_0, z_0)$と法線ベクトル$\boldsymbol{n} = (a, b, c)$が与えられたとき、平面上の任意の点$\boldsymbol{r} = (x, y, z)$は
  $\boldsymbol{n} \cdot (\boldsymbol{r} - \boldsymbol{p}) = 0$を満たす。よって
  \begin{equation*}
    a(x - x_0) + b(y - y_0) + c(z - z_0) = 0
  \end{equation*}

  となる。これを整理すると、
  \begin{equation*}
    ax + by + cz = a x_0 + b y_0 + c z_0 = d
  \end{equation*}
   となり、これが平面の方程式となる。
\end{theorembox}

それでは問題を解いてみよう。

\begin{problem}
  点$(0, 2, 1)$を通り、$\boldsymbol{n} = (1, -2, 3)$を法線ベクトルとする平面の方程式を求めよ。
\end{problem}

\begin{problem}
  $xyz$平面の平面$P: 2x - y + 3z = 1$に関して、点$A \begin{pmatrix} 1 \\ 2 \\ 4 \end{pmatrix}$と対照な
  点$A'$を求めよ。

  \dotfill
  解答 \\
\end{problem}

\section{連立方程式と階数}
\begin{definitionbox}[階数]
  任意の行列$A$は行基本変形を繰り返すことによって、\textbf{階段行列}にすることができる。このとき、この階段行列
  のなかの少なくとも1つは0ではない成分を持つ行の個数$r$を行列$A$の\textbf{階数}といい、$\text{rank} A$で表す。

  任意の行基本変形に対して、行列$A$の階数は一意に定まる。
\end{definitionbox}

連立方程式の解と階数に関する定理を紹介する。

\begin{theorembox}[連立方程式と階数]
  $n$元連立一次方程式に関して、$\boldsymbol{A}$を係数行列、$\boldsymbol{b}$を定数ベクトルとして$\boldsymbol{A} \boldsymbol{b}$が拡大係数行列とする。
  このとき、次のことが成り立つ。
  
  \begin{itemize}
    \item rank $\boldsymbol{A} = $ rank $\boldsymbol{A} \boldsymbol{b} = r$のとき、連立方程式は解を持つ。
    \item $n > \text{rank} A$なら、連立方程式は無数の解を持つ。
    \item $n = $\text{rank} $\boldsymbol{A} = $ rank $\boldsymbol{A} \boldsymbol{b} = r$のとき、連立方程式はただ1つの解を持つ。
    \item rank $\boldsymbol{A} = $ rank $\boldsymbol{A} \boldsymbol{b} = r < n$のとき、連立方程式は解を持たない。
  \end{itemize}
\end{theorembox}

次に、行列の世界の積の逆元を意味する\textbf{逆行列}について紹介する。

\begin{definitionbox}[逆行列]
  正方行列$A$に対して、$A$と積をとると単位行列$I$が得られるとき、$A$の逆行列$A^{-1}$が存在するといい、
  $A A^{-1} = A^{-1} A = I$を満たす。 逆行列が存在するとき、$A$は\textbf{正則行列}であるという。
\end{definitionbox}

\begin{theorembox}[逆行列と余因子行列]
  $n$次正方行列$A$が正則であるとき、その逆行列$A^{-1}$は次のように表される。
  
  \begin{equation*}
    A^{-1} = \frac{1}{\det A} \tilde{A}
  \end{equation*}

  $\tilde{A}$は$A$の余因子行列であり、$A$の余因子$C_{ij}$は次のように表される。
  \begin{equation*}
    C_{ij} = \tilde{A_{ji}}
  \end{equation*}
  ここで、$\tilde{A_{ji}}$は$A$の$i$行$j$列を取り除いた行列の行列式に$(-1)^{i+j}$をかけたものである。
\end{theorembox}

逆行列に関する問題を解いてみよう。\textbf{掃き出し法}によって逆行列を求める問題を解いてみよう。

\begin{problem}
  \begin{equation*}
    A = \begin{pmatrix}
      2 & 1 & 1 \\
      1 & 1 & 1 \\
      -2 & 0 & 1
    \end{pmatrix}
  \end{equation*}
  \dotfill

  解答 \\
  掃き出し法とは、$A \boldsymbol{X} = \boldsymbol{I}$の形になるように、
  \begin{equation*}
    \begin{pmatrix}
      A & I
    \end{pmatrix} \to \begin{pmatrix}
      I & A^{-1}
    \end{pmatrix}
  \end{equation*}
  ように、行列$A$を$I$に変換する方法である。この方法を用いて、
  \begin{equation*}
    \begin{pmatrix}
      2 & 1 & 1 & 1 & 0 & 0 \\
      1 & 1 & 1 & 0 & 1 & 0 \\
      -2 & 0 & 1 & 0 & 0 & 1
    \end{pmatrix} \to \begin{pmatrix}
      1 & 0 & 0 & 1 & -1 & 0 \\
      0 & 1 & 0 & -3 & 4 & -1 \\
      0 & 0 & 1 & 2 & -2 & 1
    \end{pmatrix}
  \end{equation*}
\end{problem}

\section{線形写像}

\subsection{線形写像}
\begin{definitionbox}[線形写像]
  $\forall \boldsymbol{x}, \boldsymbol{y} \in \mathbb{R}^n$, $\forall \lambda \in \mathbb{R}$に対して、次の2つの条件を満たす写像$f: \mathbb{R}^n \to \mathbb{R}^m$を\textbf{線形写像}という。

  \begin{enumerate}
    \item $f(\boldsymbol{x} + \boldsymbol{y}) = f(\boldsymbol{x}) + f(\boldsymbol{y})$
    \item $f(\lambda \boldsymbol{x}) = \lambda f(\boldsymbol{x})$
  \end{enumerate}
\end{definitionbox}

\subsection{表現行列}

\begin{theorembox}[表現行列]
  どんな線形写像$f: \mathbb{R}^n \to \mathbb{R}^m$に対しても、ある一意な行列$A$が存在して、
  \begin{equation*}
    f(\boldsymbol{x}) = A \boldsymbol{x}
  \end{equation*}
  と表される。この行列$A$を$f$の\textbf{表現行列}という。
\end{theorembox}

表現行列に関わる問題を解いてみよう。

\begin{problem}
  線形写像$f: \mathbb{R}^2 \to \mathbb{R}^2$が次のように定義されるとき、$f$の表現行列を求めよ。
  \begin{equation*}
    \text{fは}y = (\tan x) \text{での鏡映}
  \end{equation*}
  \dotfill

  解答 \\
  線形写像の表現行列を求めるには、以下の2つの方法がある。
  \begin{itemize}
    \item 定義域の写像の元を一般的に表して、終域の元を求める。
    \item 基底を用いて表現行列を求める。
  \end{itemize}

  今回は2の方法を用いて表現行列を求める。$xy$座標では$(1, 0)$と$(0, 1)$を基底として用いることができるので、
  それぞれの写像$f$による像を求める。

  \begin{equation*}
    f(1, 0) = (\tan 1, 0) = (\cos 2 x, \sin 2 x)
  \end{equation*}
  \begin{equation*}
    f(0, 1) = (0, \tan 1) = (\sin 2 x, -\cos 2 x)
  \end{equation*}
  よって求める表現行列は、
  \begin{equation*}
    A = \begin{pmatrix}
      \cos 2 x & \sin 2 x \\
      \sin 2 x & -\cos 2 x
    \end{pmatrix}
  \end{equation*}
\end{problem}

\begin{problem}
  $\mathbb{R}^3$の点$A$を、$A$から平面$x - 2y + z = 0$へ下ろした垂線の足に写す線形写像の表現行列を求めよ。
  \dotfill

  解答 \\
  $\mathbb{R}^3$の単位ベクトルがどのように写るかを考える。$\boldsymbol{e}_1 = (1, 0, 0)$に対して
  法線ベクトル$\boldsymbol{n} = (1, -2, 1)$を用いて、$\boldsymbol{e}_1$から平面へ下ろした垂線の足は
  \begin{equation*}
    \boldsymbol{n} + t \boldsymbol{e}_1 = (1 + t, -2t, t)
  \end{equation*}
  これが平面$x - 2y + z = 0$上にあればよいので、$t = -\frac{1}{6}$となる。どうの様にすることで、
  $\boldsymbol{e}_2 = (0, 1, 0)$と$\boldsymbol{e}_3 = (0, 0, 1)$が写るかを考えると、求める表現行列は

  \begin{equation*}
    A = \begin{pmatrix}
      \frac{5}{6} & \frac{1}{3} & -\frac{1}{6} \\
      -\frac{1}{3} & \frac{1}{3} & \frac{1}{3} \\
      -\frac{1}{6} & \frac{1}{3} & \frac{5}{6}
    \end{pmatrix}
  \end{equation*}
\end{problem}

\section{部分空間}
\begin{definitionbox}[部分空間]
  ベクトル空間$V$の部分集合$W$が次の2つの条件を満たすとき、$W$は$V$の\textbf{部分空間}であるという。

  \begin{enumerate}
    \item $\boldsymbol{0} \in W$
    \item $\boldsymbol{u}, \boldsymbol{v} \in W$に対して、$\boldsymbol{u} + \boldsymbol{v} \in W$
    \item $\lambda \boldsymbol{u} \in W$
  \end{enumerate}
\end{definitionbox}

部分空間を扱った問題を解いてみよう。

\begin{problem}
  次の$W$が$\mathbb{R}^3$の部分空間となるか判定せよ。
  \begin{equation*}
    W = \{ (x, y, z) \in \mathbb{R}^3 | 3x - z = y + 2z = x - y \}
  \end{equation*}

  \dotfill

  解答 \\
  まず、$\boldsymbol{0} = (0, 0, 0) \in W$である。
\end{problem}

\section{基底と次元}
\begin{definitionbox}[基底]
  ベクトル空間$V$のベクトルたち$\boldsymbol{v}_1, \boldsymbol{v}_2, \ldots, \boldsymbol{v}_n$が次の2つの条件を満たすとき、これらは$V$の\textbf{基底}をなすという。

  \begin{enumerate}
    \item $\boldsymbol{v}_1, \boldsymbol{v}_2, \ldots, \boldsymbol{v}_n$は$V$の1次独立である。
    \item $V$の任意のベクトル$\boldsymbol{v}$は$\boldsymbol{v}_1, \boldsymbol{v}_2, \ldots, \boldsymbol{v}_n$の線形結合で表される。
  \end{enumerate}
\end{definitionbox}

\begin{definitionbox}[次元]
  ベクトル空間$V$の基底の数を$V$の\textbf{次元}といい、$\dim V$で表す。
  $V$が有限個のベクトルで生成できないとき、$V$は無限次元であるという($\dim V = \infty$)。
\end{definitionbox}

\begin{definitionbox}[一次独立と一次従属]
  ベクトル$\boldsymbol{v}_1, \boldsymbol{v}_2, \ldots, \boldsymbol{v}_n$が\textbf{一次独立}であるとは、
  $\lambda_1, \lambda_2, \cdots, \lambda_n \in \mathbb{R}$に対して、
  \begin{equation*}
    \lambda_1 \boldsymbol{v}_1 + \lambda_2 \boldsymbol{v}_2 + \cdots + \lambda_n \boldsymbol{v}_n = \boldsymbol{0}
  \end{equation*}
  が成り立つとき、$\lambda_1 = \lambda_2 = \cdots = \lambda_n = 0$に限るをいう。

  それ以外のとき、$\boldsymbol{v}_1, \boldsymbol{v}_2, \ldots, \boldsymbol{v}_n$は\textbf{一次従属}であるという。
\end{definitionbox}

一次独立について重要な定理を紹介する。

\begin{theorembox}[一次独立と同値な条件]
  ベクトル$\boldsymbol{v}_1, \boldsymbol{v}_2, \ldots, \boldsymbol{v}_n$が一次独立であるための必要十分条件は、以下のいずれかが成り立つことである。
  $V= \mathbb{R}^n$のとき、

  \begin{itemize}
    \item $\boldsymbol{v}_1, \boldsymbol{v}_2, \ldots, \boldsymbol{v}_n \in \mathbb{R}^n$に対し、$n \times k$行列$A = (\boldsymbol{v}_1, \boldsymbol{v}_2, \ldots, \boldsymbol{v}_n)$のランクが$k$である。
    \item $\det A \neq 0$
    \item $A$が正則行列である。
    \item $<\boldsymbol{v}_1, \boldsymbol{v}_2, \ldots, \boldsymbol{v}_n> = \mathbb{R}^n$
  \end{itemize}
\end{theorembox}
\begin{problem}
  次のベクトルの組みが一次独立であるか判定せよ。
  \begin{equation*}
    \boldsymbol{a}_1 = \begin{pmatrix} 2 \\ 3 \\ -1  \\ 1 \end{pmatrix}, \boldsymbol{a}_2 = \begin{pmatrix} -3 \\ 2 \\ 0 \\ -2 \end{pmatrix}, \boldsymbol{a}_3 = \begin{pmatrix} 1 \\ -5 \\ -1 \\ 1 \end{pmatrix}
  \end{equation*}

  \dotfill

  解答 \\
  行列
  \begin{equation*}
    A = \begin{pmatrix}
      2 & -3 & 1 \\
      3 & 2 & -5 \\
      -1 & 0 & -1 \\
      1 & -2 & 1
    \end{pmatrix}
  \end{equation*}
  に対して行基本変形をして簡約化して階段行列を求めると、
  \begin{equation*}
    \begin{pmatrix}
      1 & 0 & 1 \\
      0 & 1 & 0 \\
      0 & 0 & 1 \\
      0 & 0 & 0
    \end{pmatrix}
  \end{equation*}
  となるので、$rank A = 3$であり、$\boldsymbol{a}_1, \boldsymbol{a}_2, \boldsymbol{a}_3$は一次独立である。
\end{problem}


\begin{definitionbox}[基底の変換行列]
  $V$を$\mathbb{R}$上のベクトル空間とし、$\dim V = n$とする。$V$の2つの基底
  \begin{equation*}
   A:  \boldsymbol{u}_1, \boldsymbol{u}_2, \ldots, \boldsymbol{u}_n
  \end{equation*}
  \begin{equation*}
    B: \boldsymbol{v}_1, \boldsymbol{v}_2, \ldots, \boldsymbol{v}_n
  \end{equation*}
  が与えられたとき、これらの基底の間の変換行列$P$は次のように定義される。

  \begin{equation*}
    \begin{pmatrix}
      \boldsymbol{v}_1 & \boldsymbol{v}_2 & \cdots & \boldsymbol{v}_n
    \end{pmatrix} = 
    \begin{pmatrix}
      \boldsymbol{u}_1 & \boldsymbol{u}_2 & \cdots & \boldsymbol{u}_n
    \end{pmatrix}
    \begin{pmatrix}
      p_{11} & p_{12} & \cdots & p_{1n} \\
      p_{21} & p_{22} & \cdots & p_{2n} \\
      \vdots & \vdots & \ddots & \vdots \\
      p_{n1} & p_{n2} & \cdots & p_{nn}
    \end{pmatrix}
  \end{equation*}
  行列$P$を$A$から$B$への基底の変換行列という。
\end{definitionbox}

\begin{problem}
  \begin{equation*}
    \boldsymbol{a}_1 = \begin{pmatrix}
      2 \\ 0 \\ 1
    \end{pmatrix}, \boldsymbol{a}_2 = \begin{pmatrix}
      1 \\ 1 \\ 0
    \end{pmatrix}, \boldsymbol{a}_3 = \begin{pmatrix}
      0 \\ -1 \\ 1
    \end{pmatrix}
  \end{equation*}
  
  \begin{equation*}
    \boldsymbol{b}_1 = \begin{pmatrix}
      1 \\ -1 \\ 2
    \end{pmatrix}, \boldsymbol{b}_2 = \begin{pmatrix}
      2 \\ 2 \\ -1
    \end{pmatrix}, \boldsymbol{b}_3 = \begin{pmatrix}
      0 \\ 1 \\ -1
    \end{pmatrix}
  \end{equation*}
  として、$\mathbb{R}^3$ 
  の基底$A = \{ \boldsymbol{a}_1, \boldsymbol{a}_2, \boldsymbol{a}_3 \}$から$\mathbb{R}^3$の基底$B = \{ \boldsymbol{b}_1, \boldsymbol{b}_2, \boldsymbol{b}_3 \}$への変換行列$P$を求めよ。

  \dotfill

  解答 \\
  定義より、$(\boldsymbol{b}_1, \boldsymbol{b}_2, \boldsymbol{b}_3) = ( \boldsymbol{a}_1, \boldsymbol{a}_2, \boldsymbol{a}_3) P$であるので、
  $A = (\boldsymbol{a}_1, \boldsymbol{a}_2, \boldsymbol{a}_3)$, $B = (\boldsymbol{b}_1, \boldsymbol{b}_2, \boldsymbol{b}_3)$として、
  $P = A^{-1} B$を求める。
  \begin{equation*}
    A^{-1} = \begin{pmatrix}
      1 & -1 & -1 \\
      -1 & 2 & 2 \\
      -1 & 1 & 2
    \end{pmatrix}
  \end{equation*}
  となるので、
  \begin{equation*}
    P = A^{-1} B = \begin{pmatrix}
      0 & 1 & 0 \\
      1 & 0 & 0 \\
      2 & -2 & -1
    \end{pmatrix}
  \end{equation*}

  別解 \\
  掃き出し法を用いて直接$P = A^{-1} B$を求めることもできる。($(A B)$の形になるように行列を並べて、$A$の部分を単位行列に変換する)

  \begin{equation*}
    \begin{pmatrix}
      2 & 1 & 0 & 1 & 2 & 0 \\
      0 & 1 & -1 & -1 & 2 & 1 \\
      1 & 0 & 1 & 0 & -1 & -1
    \end{pmatrix} \to \begin{pmatrix}
      1 & 0 & 0 & 0 & 1 & 0 \\
      0 & 1 & 0 & 0 & 1 & 1 \\
      0 & 0 & 1 & 0 & -1 & -1
    \end{pmatrix}
  \end{equation*}
\end{problem}

参考
\begin{itemize}
  \item \url{https://uxhpu.net/mathematics/change_of_basis/}
\end{itemize}

\end{document}