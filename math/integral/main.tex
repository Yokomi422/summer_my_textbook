\documentclass{jlreq}

\usepackage{titlesec}
\usepackage{listings}
\usepackage{fancyhdr}

% \adjustbox
\usepackage{adjustbox}

% tcolorboxの設定
\usepackage[most]{tcolorbox} 
\tcbuselibrary{breakable}
\tcbuselibrary{skins}
\tcbuselibrary{listingsutf8}
% タイトルのフォーマットを変更
\titleformat{\title}
  {\centering\Huge\bfseries}
  {}
  {0em} 
  {}

\titleformat{\subtitle}
  {\centering\Large\itshape}
  {}
  {0em}
  {}

\titleformat{\subsubsection}[block]
  {\normalfont\normalsize\bfseries}
  {\arabic{subsubsection}.}
  {1em}
  {}

\titleformat{\section}[block]
  {\normalfont\large\bfseries}
  {\Roman{section}.}
  {1em} 
  {}
  [\titleline{\titlerule[1pt]}]

\titleformat{\subsection}[block]
  {\normalfont\normalsize\bfseries}
  {\roman{subsection}.}
  {1em}
  {}

% listingsの設定

\renewcommand{\lstlistingname}{コード}

\lstset{
	breaklines = true,
	language = Python,
	keywordstyle = {\bfseries \color[cmyk]{0,1,0,0}},
	commentstyle = {\itshape \color[cmyk]{1,0.4,1,0}},
	numbers = left,
	numberstyle = \tiny,
	stepnumber = 1,
	% frameとnumberの間の距離
	numbersep = 10pt,
	frame = single,
	basicstyle = \ttfamily,
	tabsize = 2,
	captionpos = t,
	backgroundcolor={\color[gray]{.90}},
	showstringspaces = false,
}

% headerの設定
\pagestyle{fancy}
\fancyhf{}

\fancyhead[RO,RE]{\rightmark}
\fancyhead[LO,LE]{\leftmark} 
\fancyfoot[C]{\thepage}

% tikzの設定
\usepackage{tikz}

\begin{document}

\section{三角関数}

\begin{tcolorbox}[enhanced,
  colback=white!85!gray,
  drop fuzzy shadow,
  boxrule=0.3mm,
  arc=0mm,
  left=0pt,
  top=0pt,
  sharp corners,
  width=\textwidth,
  ]
  \textbf{問題1}
  以下の不定積分を求めよ。
  \begin{equation*}
    \int \tan ^3 x  dx
  \end{equation*}
\tcblower

\begin{tcolorbox}[
  coltext=white!10!blue,
  colback=white!90!purple!90!blue,
  drop fuzzy shadow,
  boxrule=0mm,
  arc=0mm,
  width=1.3cm,
  left=0pt,
  right=0pt,
  top=0pt,
  bottom=0pt,
  halign=flush left,
]
\end{tcolorbox}
\tcblower
\textbf{回答:}


\begin{tcolorbox}[
  coltext=black,
  colback=yellow!50!white,
  drop fuzzy shadow,
  boxrule=0.4mm,
  arc=3mm,
  left=1mm,
  right=1mm,
  top=1mm,
  bottom=1mm,
  title=ポイント,
  sharp corners,
  width=\textwidth-2mm,
  ]
  \begin{itemize}
    \item 三角関数は2乗の形に強い
    \item $\tan x^2 + 1 = \frac{1}{\cos ^2}の利用$
    \item 接触型の積分
  \end{itemize}
\end{tcolorbox}

\begin{align*}
  \int \tan ^3 x dx &= \int \tan x \tan ^2 x dx \\
  &= \int \tan x (\frac{1}{\cos^2 x} - 1) dx \\
  &= \int \tan x \frac{1}{\cos^2 x} dx - \int \tan x dx \\
  &= \int \tan x \frac{1}{\cos^2 x} dx + \int \frac{-\sin x}{\cos x} dx \\
  &= \int \tan x (\frac{d}{dx} \tan x) dx + \int (\frac{d}{dx} \cos x) \frac{1}{\cos x} dx \\
  &= \frac{1}{2} \tan ^2 x + \log |\cos x| + C
\end{align*}
\end{tcolorbox}% タイトルの独立

%%%%%%%%%%%%%%%%%%%%%%%%%%%%%%%%%%%%%%%%%%%%%%%%%%%%%%
\begin{tcolorbox}[enhanced,
  colback=white!85!gray,
  drop fuzzy shadow,
  boxrule=0.3mm,
  arc=0mm,
  left=0pt,
  top=0pt,
  sharp corners,
  width=\textwidth,
  ]
  \textbf{問題2}
  以下の定積分を求めよ。
  \begin{equation*}
    \int_0^{2 \pi} \sqrt{1 + \cos x} dx
  \end{equation*}
\tcblower

\begin{tcolorbox}[
  coltext=white!10!blue,
  colback=white!90!purple!90!blue,
  drop fuzzy shadow,
  boxrule=0mm,
  arc=0mm,
  width=1.3cm,
  left=0pt,
  right=0pt,
  top=0pt,
  bottom=0pt,
  halign=flush left,
]
\end{tcolorbox}
\tcblower
\textbf{回答:}


\begin{tcolorbox}[
  coltext=black,
  colback=yellow!50!white,
  drop fuzzy shadow,
  boxrule=0.4mm,
  arc=3mm,
  left=1mm,
  right=1mm,
  top=1mm,
  bottom=1mm,
  title=ポイント,
  sharp corners,
  width=\textwidth-2mm,
  ]
  \begin{itemize}
    \item 根号 \rightarrow 外す

    根号を外すパターンは以下のようなものがある。
    \begin{itemize}
      \item 置換積分
      \item 三角関数の場合は、三角関数の公式を利用
    \end{itemize}
  \end{itemize}
\end{tcolorbox}
$\cos 2 x = \cos ^2 x - 1$より、
\begin{align*}
\end{align*}
\end{tcolorbox}
%%%%%%%%%%%%%%%%%%%%%%%%%%%%%%%%%%%%%%%%%%%%%%%%%%%%%%

\end{document}
