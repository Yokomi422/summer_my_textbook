\section{その他頻出アルゴリズム}
\subsection{bit全探索}
bit全探索とは、選ぶ選ばないの2択を全ての組み合わせで試すアルゴリズムです。例えば、$N$個の要素があるとき、それぞれの要素を選ぶか選ばないかの2択を$2^N$通り全て試すことができます。

部分和問題を使ってbit全探索の概要を説明します。部分和問題とは、$N$個の整数$a_1, a_2, \ldots, a_N$が与えられたとき、それらの整数の中からいくつか選んで総和を$K$とすることができるかを判定する問題です。

$N = 3, K = 10, a = \{1, 4, 5\}$のとき、$a$のそれぞれの要素を使うか使わないかの2択、全部で$2^3 = 8$通りの組み合わせがあります。それぞれの組み合わせに対して、総和が$K$となるかを判定します。

\begin{table}[h]
  \centering
  \begin{tabular}{|c|c|c|c|c|}
    \hline
    $1$ & $4$ & $5$ & 総和 & 判定 \\
    \hline
    0 & 0 & 0 & 0 & No \\
    0 & 0 & 1 & 5 & No \\
    0 & 1 & 0 & 4 & No \\
    0 & 1 & 1 & 9 & No \\
    1 & 0 & 0 & 1 & No \\
    1 & 0 & 1 & 6 & No \\
    1 & 1 & 0 & 5 & No \\
    1 & 1 & 1 & 10 & Yes \\
    \hline
  \end{tabular}
\end{table}

すべての場合を列挙してしまえば、あとは簡単に処理できそうですね。ではどのようにして列挙すればよいでしょうか。Pythonの標準ライブラリの
itertools.productを使った方法や、bit演算を使った方法があります。両方で部分和問題を解いてみましょう。

\begin{lstlisting}[caption=itertools.productを使ったbit全探索, label=bit_search, frame=TRBL]
from itertools import product

def bit_search(array: list[int], value: int) -> bool:
      size = len(array)
      flag = False
      
      for prod in product([0, 1], repeat=size):
          total = 0
          for index, to_use in enumerate(prod):
              if to_use:
                  total += array[index]
          
          if total == value:
              flag = True
      
      return flag
  
  def main():
      n, v = map(int, input().split())
      A = list(map(int, input().split()))
      
      is_ok = bit_search(A, v)
      
      print("Yes" if is_ok else "No")
                  
  
  if __name__ == "__main__":
      main()
  
\end{lstlisting}

itertools.productを使った実装は直感的に実装できましたが、この実装はどの言語でもできるわけではないので、bit演算を使った実装を紹介します。
bit演算は0と1からなる2進数の数の各桁に対して、配列の要素をマッピングして0なら使わない、1なら使うという処理を行います。
$N = 3, K = 10, a = \{1, 4, 5\}$のとき対応する2進数を考えると以下のようになります。0と1の組み合わせをすべて列挙すればいいです。
bit演算という処理を使うことで、$2^N$通りの組み合わせを簡単に列挙することができます。

\begin{table}[h]
  \centering
  \begin{tabular}{|c|c|c|c|c|}
    \hline
    $1$ & $4$ & $5$ & 総和 & 判定 \\
    \hline
    0 & 0 & 0 & 0 & No \\
    0 & 0 & 1 & 5 & No \\
    0 & 1 & 0 & 4 & No \\
    0 & 1 & 1 & 9 & No \\
    1 & 0 & 0 & 1 & No \\
    1 & 0 & 1 & 6 & No \\
    1 & 1 & 0 & 5 & No \\
    1 & 1 & 1 & 10 & Yes \\
    \hline
  \end{tabular}
\end{table}

bit演算を理解するには、bitシフトや、ANDやORなどのbit演算子を理解する必要があります。bit演算を使った実装は以下のようになります。
\texttt{1 << n}は$2^n$を表すbitシフトです。$2^n$通りの組み合わせを列挙するために、$2^N$を計算しています。
\texttt{bit \& (1 << i)}は、\texttt{bit}の\texttt{i}番目のbitが立っているかどうかを判定しています。立っている場合は\texttt{A[i]}を総和に加えます。
例えば、bitが010010であるとき、bit \& (1 < 2)とすれば1を2だけ左シフトした100と010010をANDするので、010010に右から3番目のbitが立っているかどうかを判定できます。
bitの位置と配列の要素の位置のマッピングを適切に行えば、部分和問題を解くことができます。下のコードではbitの右の桁から順にAの元の要素に対応させています。

\begin{lstlisting}[caption=bit演算を使ったbit全探索, label=bit_search, frame=TRBL]
def main():
  n, v = map(int, input().split())
  A = list(map(int, input().split()))
  
  for bit in (1 << n):
      total = 0
      for i in range(n):
          if bit & (1 << i):
              total += A[i]
      
      if total == v:
          print("Yes")
          exit()
  
  print("No")
              

if __name__ == "__main__":
  main()
\end{lstlisting}

\section{3つ以上の全列挙(再帰による全列挙)}

\section{問題}

\textbf{問題1} AtCoder Beginner Contest 214 B How many?\\
\textbf{問題2}  AtCoder Beginner Contest 367 C Enumerate Sequences \\


\section{参考}
\textbf{bit全探索}
\begin{itemize}
	\item \url{https://qiita.com/u2dayo/items/68e35815659b1041c3c2}
	\item \url{https://algo-method.com/tasks/1131I9eL}
\end{itemize}

