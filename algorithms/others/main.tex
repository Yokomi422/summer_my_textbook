\documentclass{jlreq}

\usepackage{titlesec}
\usepackage{listings}
\usepackage{fancyhdr}

% \adjustbox
\usepackage{adjustbox}

% tcolorboxの設定
\usepackage[most]{tcolorbox} 
\tcbuselibrary{breakable}
\tcbuselibrary{skins}
\tcbuselibrary{listingsutf8}
% タイトルのフォーマットを変更
\titleformat{\title}
  {\centering\Huge\bfseries}
  {}
  {0em} 
  {}

\titleformat{\subtitle}
  {\centering\Large\itshape}
  {}
  {0em}
  {}

\titleformat{\subsubsection}[block]
  {\normalfont\normalsize\bfseries}
  {\arabic{subsubsection}.}
  {1em}
  {}

\titleformat{\section}[block]
  {\normalfont\large\bfseries}
  {\Roman{section}.}
  {1em} 
  {}
  [\titleline{\titlerule[1pt]}]

\titleformat{\subsection}[block]
  {\normalfont\normalsize\bfseries}
  {\roman{subsection}.}
  {1em}
  {}

% listingsの設定

\renewcommand{\lstlistingname}{コード}

\lstset{
	breaklines = true,
	language = Python,
	keywordstyle = {\bfseries \color[cmyk]{0,1,0,0}},
	commentstyle = {\itshape \color[cmyk]{1,0.4,1,0}},
	numbers = left,
	numberstyle = \tiny,
	stepnumber = 1,
	% frameとnumberの間の距離
	numbersep = 10pt,
	frame = single,
	basicstyle = \ttfamily,
	tabsize = 2,
	captionpos = t,
	backgroundcolor={\color[gray]{.90}},
	showstringspaces = false,
}

% headerの設定
\pagestyle{fancy}
\fancyhf{}

\fancyhead[RO,RE]{\rightmark}
\fancyhead[LO,LE]{\leftmark} 
\fancyfoot[C]{\thepage}

% tikzの設定
\usepackage{tikz}

\usepackage{amsmath}
\usepackage{url}

% 問題用のボックス環境
\newtcolorbox{problem}[1][]{enhanced,
  colback=white!85!gray,
  drop fuzzy shadow,
  boxrule=0.3mm,
  arc=0mm,
  left=0pt,
  top=0pt,
  sharp corners,
  width=\textwidth,
  title=\textbf{問題},
  #1
}

\usepackage{url}

\begin{document}
\section{その他頻出アルゴリズム}
\subsection{bit全探索}
bit全探索とは、選ぶ選ばないの2択を全ての組み合わせで試すアルゴリズムです。例えば、$N$個の要素があるとき、それぞれの要素を選ぶか選ばないかの2択を$2^N$通り全て試すことができます。

部分和問題を使ってbit全探索の概要を説明します。部分和問題とは、$N$個の整数$a_1, a_2, \ldots, a_N$が与えられたとき、それらの整数の中からいくつか選んで総和を$K$とすることができるかを判定する問題です。

$N = 3, K = 10, a = \{1, 4, 5\}$のとき、$a$のそれぞれの要素を使うか使わないかの2択、全部で$2^3 = 8$通りの組み合わせがあります。それぞれの組み合わせに対して、総和が$K$となるかを判定します。

\begin{table}[h]
  \centering
  \begin{tabular}{|c|c|c|c|c|}
    \hline
    $1$ & $4$ & $5$ & 総和 & 判定 \\
    \hline
    0 & 0 & 0 & 0 & No \\
    0 & 0 & 1 & 5 & No \\
    0 & 1 & 0 & 4 & No \\
    0 & 1 & 1 & 9 & No \\
    1 & 0 & 0 & 1 & No \\
    1 & 0 & 1 & 6 & No \\
    1 & 1 & 0 & 5 & No \\
    1 & 1 & 1 & 10 & Yes \\
    \hline
  \end{tabular}
\end{table}

すべての場合を列挙してしまえば、あとは簡単に処理できそうですね。ではどのようにして列挙すればよいでしょうか。Pythonの標準ライブラリの
itertools.productを使った方法や、bit演算を使った方法があります。両方で部分和問題を解いてみましょう。

\begin{lstlisting}[caption=itertools.productを使ったbit全探索, label=bit_search, frame=TRBL]
from itertools import product

def bit_search(array: list[int], value: int) -> bool:
      size = len(array)
      flag = False
      
      for prod in product([0, 1], repeat=size):
          total = 0
          for index, to_use in enumerate(prod):
              if to_use:
                  total += array[index]
          
          if total == value:
              flag = True
      
      return flag
  
  def main():
      n, v = map(int, input().split())
      A = list(map(int, input().split()))
      
      is_ok = bit_search(A, v)
      
      print("Yes" if is_ok else "No")
                  
  
  if __name__ == "__main__":
      main()
  
\end{lstlisting}

itertools.productを使った実装は直感的に実装できましたが、この実装はどの言語でもできるわけではないので、bit演算を使った実装を紹介します。
bit演算は0と1からなる2進数の数の各桁に対して、配列の要素をマッピングして0なら使わない、1なら使うという処理を行います。
$N = 3, K = 10, a = \{1, 4, 5\}$のとき対応する2進数を考えると以下のようになります。0と1の組み合わせをすべて列挙すればいいです。
bit演算という処理を使うことで、$2^N$通りの組み合わせを簡単に列挙することができます。

\begin{table}[h]
  \centering
  \begin{tabular}{|c|c|c|c|c|}
    \hline
    $1$ & $4$ & $5$ & 総和 & 判定 \\
    \hline
    0 & 0 & 0 & 0 & No \\
    0 & 0 & 1 & 5 & No \\
    0 & 1 & 0 & 4 & No \\
    0 & 1 & 1 & 9 & No \\
    1 & 0 & 0 & 1 & No \\
    1 & 0 & 1 & 6 & No \\
    1 & 1 & 0 & 5 & No \\
    1 & 1 & 1 & 10 & Yes \\
    \hline
  \end{tabular}
\end{table}

bit演算を理解するには、bitシフトや、ANDやORなどのbit演算子を理解する必要があります。bit演算を使った実装は以下のようになります。
\texttt{1 << n}は$2^n$を表すbitシフトです。$2^n$通りの組み合わせを列挙するために、$2^N$を計算しています。
\texttt{bit \& (1 << i)}は、\texttt{bit}の\texttt{i}番目のbitが立っているかどうかを判定しています。立っている場合は\texttt{A[i]}を総和に加えます。
例えば、bitが010010であるとき、bit \& (1 < 2)とすれば1を2だけ左シフトした100と010010をANDするので、010010に右から3番目のbitが立っているかどうかを判定できます。
bitの位置と配列の要素の位置のマッピングを適切に行えば、部分和問題を解くことができます。下のコードではbitの右の桁から順にAの元の要素に対応させています。

\begin{lstlisting}[caption=bit演算を使ったbit全探索, label=bit_search, frame=TRBL]
def main():
  n, v = map(int, input().split())
  A = list(map(int, input().split()))
  
  for bit in (1 << n):
      total = 0
      for i in range(n):
          if bit & (1 << i):
              total += A[i]
      
      if total == v:
          print("Yes")
          exit()
  
  print("No")
              

if __name__ == "__main__":
  main()
\end{lstlisting}

\textbf{参考}
\begin{itemize}
	\item \url{https://qiita.com/u2dayo/items/68e35815659b1041c3c2}
	\item \url{https://algo-method.com/tasks/1131I9eL}
\end{itemize}

\subsection{3つ以上の全列挙(再帰による全列挙)}
bit全探索の応用としてより一般的な複数選択肢があるような問題を扱います。AtCoder Beginner Contest 367 C Enumerate Sequences
を例題にして配列Rが与えられ$R_i$に関して$1 \leq i \leq R_i$の範囲で全列挙を行い、合計がKの倍数になるような組み合わせを辞書順に求める問題です。

解法はitertool.productを使った方法と、再帰関数を使った方法があります。

\begin{lstlisting}[caption=itertools.productを使った解法, label=product_ans, frame=TRBL]
from itertools import product

  def main():
      n, k = map(int, input().split())
      R = list(map(int, input().split()))
      
      values = [[] for _ in range(n)]
      for i in range(n):
          values[i] = list(range(1, R[i] + 1))
      
      for prod in product(*values):
          total = 0
          for value in prod:
              total += value
          if total % k == 0:
              print(*prod)
  
  if __name__ == "__main__":
      main()
\end{lstlisting}


\begin{lstlisting}[caption=再帰関数を使った解法, label=recursive_ans, frame=TRBL]
def generate_combinations(n, values, current_combination, index, k):
  if index == n:
      total = sum(current_combination)
      if total % k == 0:
          print(*current_combination)
      return

  for value in values[index]:
      generate_combinations(n, values, current_combination + [value], index + 1, k)


def main():
  n, k = map(int, input().split())
  R = list(map(int, input().split()))

  values = [list(range(1, R[i] + 1)) for i in range(n)]

  generate_combinations(n, values, [], 0, k)


if __name__ == "__main__":
  main()
\end{lstlisting}

\subsubsection{問題}

\textbf{問題1} AtCoder Beginner Contest 214 B How many?\\
\textbf{問題2}  AtCoder Beginner Contest 367 C Enumerate Sequences \\


\textbf{参考}
\begin{itemize}
  \item \url{https://drken1215.hatenablog.com/entry/2020/01/05/185000}
\end{itemize}
\end{document}